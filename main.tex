\documentclass[a4paper]{article}

\usepackage[english]{babel}
\usepackage[utf8]{inputenc}
\usepackage{amsmath}
\usepackage{graphicx}
\usepackage[colorinlistoftodos]{todonotes}

\title{Slavery Essay}

\author{You}

\date{\today}

\begin{document}
\maketitle
 American Slavery Essay
Slavery is one of the darkest deeds ever done in America's history, but it is important to know and learn from it so that we don't repeat our mistakes so in this essay I believe in the quote You can only protect your liberties in this world by protecting the other man's 
freedom. You can only be free if I am free. 

Slavery in America can be traced as far back as when Europeans began settling the North American continent. The first town established in the New World was Jamestown in 1607, and the first slave arrived on that continent in 1619. European pioneers who colonized North America brought slaves with them in order to help settle the new land, work their plantations growing valuable cash crops such as tobacco, wheat, cotton, and sugar, and to cook and clean in their homes. Most people didn't see slavery as a problem at this time because it was quite rare in the New World with only a few wealthy landowners who owned slaves, however, public opinion would soon be changed.
% * <gregtyka@gmail.com> 2016-05-11T19:28:01.222Z:
%
% > Slavery in America can be traced as far back as when Europeans began settling the North American continent. The first town established in the New World was Jamestown in 1607, and the first slave arrived on that continent in 1619. European pioneers who colonized North America brought slaves with them in order to help settle the new land, work their plantations growing valuable cash crops such as tobacco, wheat, cotton, and sugar, and to cook and clean in their homes. Most people didn't see slavery as a problem at this time because it was quite rare in the New World with only a few wealthy landowners who owned slaves, however, public opinion would soon be changed.
%
% This is summarized from the slavery page on brintica which can be found here  http://www.britannica.com/topic/slavery-sociology 
%
% ^.
Abolitionists (Definition being someone who deeply aposes salvary) first started appearing in America at about the time of the American revolution. Opponents of slavery included some of our distinguished Founding Fathers such as Benjamin Franklin, Alexander Hamilton, John Jay, Thomas Paine, and Benjamin Rush, who felt that slavery  infringed on the concepts of the Declaration of Independence. Most northern abolitionists were religiously inspired, such as the Quakers, and felt that slavery was a sin that must rectified immediately.
% * <gregtyka@gmail.com> 2016-05-11T19:30:09.980Z:
%
% > Abolitionists (Definition being someone who deeply aposes salvary) first started appearing in America at about the time of the American revolution. Opponents of slavery included some of our distinguished Founding Fathers such as Benjamin Franklin, Alexander Hamilton, John Jay, Thomas Paine, and Benjamin Rush, who felt that slavery  infringed on the concepts of the Declaration of Independence. Most northern abolitionists were religiously inspired, such as the Quakers, and felt that slavery was a sin that must rectified immediately.
%
% This list of abolitionist also came from brinticia found here  http://www.britannica.com/topic/abolitionism-European-and-American-social-movement and here  https://en.wikipedia.org/wiki/List_of_abolitionists
%
% ^.


The abolitionist cause was a good argument. They felt that the majority of slaves were being treated inhumanely and tortured. This disgust of southern slave-owners compelled a few abolitionists to act out in extreme measures, but the majority used peaceful protest methods. They used different methods to fight for their cause; fanatics went to the utmost of their power in killing the opposition, while others passively handed out pamphlets and flyers in protest or participated in the Underground Railroad. In This essay i will show the different abolitionists.
% * <gregtyka@gmail.com> 2016-05-11T19:32:49.498Z:
%
% > The abolitionist cause was a good argument. They felt that the majority of slaves were being treated inhumanely and tortured. This disgust of southern slave-owners compelled a few abolitionists to act out in extreme measures, but the majority used peaceful protest methods. They used different methods to fight for their cause; fanatics went to the utmost of their power in killing the opposition, while others passively handed out pamphlets and flyers in protest or participated in the Underground Railroad. In This essay i will show the different abolitionists.
%
% I wrote this with some help of grammarly and thesaurs.com 
%
% ^.

Brown was born on May 9th, 1800 in Torrington, Connecticut, and grew up in Ohio. During his adult life, Brown had trouble holding down a steady job due to business racism and charges that followed him from the 1820's and on, but by the 1850's he became deeply interested in the slavery issue.
% * <gregtyka@gmail.com> 2016-05-11T19:34:34.521Z:
%
% > Brown was born on May 9th, 1800 in Torrington, Connecticut, and grew up in Ohio. During his adult life, Brown had trouble holding down a steady job due to business racism and charges that followed him from the 1820's and on, but by the 1850's he became deeply interested in the slavery issue.
% > Brown and five of his sons became engrosed in the struggle between Pro-slavery and antislavery forces for control of the territorial government in Kansas. By the spring of 1855, Brown had assumed command of local Free-Soil militia. Within a year, pro-slavery forces had sacked the Free-Soil town of Lawrence, an event that triggered a bloody retaliation by Brown. During the night of May 24, 1856, Brown, four of his sons, and two other followers invaded the Pottawatomie River country and killed five helpless settlers, slaughtering them with sabers. Brown, who was never caught, claimed full responsibility for the act.
% > From then on, Brown became, even more, preoccupied with abolishing slavery. By 1858 he had persuaded a number of the North's most prominent abolitionists to finance his anti-slavery efforts. After protracted conspiracy, delay, and diversion, Brown finally chose Harpers Ferry as his point of attack, hoping to establish a base in the mountains to which slaves could flee. Brown assembled an armed force of 21 men about 5 miles from Harpers Ferry, and on Oct. 16, 1859, they seized the town and occupied the federal arsenal. The town was soon surrounded by local militia, and federal troops under Robert E. Lee arrived the next day. Ten of Brown's army died in the ensuing battle, and Brown himself was wounded. Arrested and charged with treason, Brown was hung on Dec. 2, 1859. William Lloyd Garrison was another abolitionist, however he did not go to the extremes that John Brown went to to free slaves. 
%
% This was summarized from my sources of Brown and gramaticly checked and fixed with thesaurs.com and grammarly here are the recources
%
% ^ <gregtyka@gmail.com> 2016-05-11T19:35:52.100Z:
%
% "John Brown Abolitionist." JOHN BROWN History. History Organization, n.d. Web. 10 May 2016 
%
% ^ <gregtyka@gmail.com> 2016-05-11T19:36:05.936Z:
%
% John Brown." Bio.com. A&E Networks Television, n.d. Web. 10 May 2016
%
% ^.

Brown and five of his sons became engrosed in the struggle between Pro-slavery and antislavery forces for control of the territorial government in Kansas. By the spring of 1855, Brown had assumed command of local Free-Soil militia. Within a year, pro-slavery forces had sacked the Free-Soil town of Lawrence, an event that triggered a bloody retaliation by Brown. During the night of May 24, 1856, Brown, four of his sons, and two other followers invaded the Pottawatomie River country and killed five helpless settlers, slaughtering them with sabers. Brown, who was never caught, claimed full responsibility for the act.

From then on, Brown became, even more, preoccupied with abolishing slavery. By 1858 he had persuaded a number of the North's most prominent abolitionists to finance his anti-slavery efforts. After protracted conspiracy, delay, and diversion, Brown finally chose Harpers Ferry as his point of attack, hoping to establish a base in the mountains to which slaves could flee. Brown assembled an armed force of 21 men about 5 miles from Harpers Ferry, and on Oct. 16, 1859, they seized the town and occupied the federal arsenal. The town was soon surrounded by local militia, and federal troops under Robert E. Lee arrived the next day. Ten of Brown's army died in the ensuing battle, and Brown himself was wounded. Arrested and charged with treason, Brown was hung on Dec. 2, 1859. William Lloyd Garrison was another abolitionist, however he did not go to the extremes that John Brown went to to free slaves. 




Born in Newburyport, Massachusetts on December 12th 1805, Garrison was seen by many as the epitome of the American abolitionist movement. Initially an advocate of moderate abolitionism while co editing Benjamin Lundy's weekly Genius of Universal Emancipation, Garrison soon began more deeply felt attacks on slavery. On January 1, 1831, he published the first issue of the Liberator, declaring slavery to be an abomination in God's sight, demanding immediate emancipation of the slaves, and vowing never to be silenced. The Liberator, in continuous weekly publication through 1865, always served as a personal release for Garrison's views on slavery, but it was also widely regarded as an authoritative voice of radical Yankee social reform in general.

In 1849, Garrison became involved in one of Boston's most notable trials of the time.Washington Goode, a black seaman had been sentenced to death for the murder of a fellow black mariner, Thomas Harding. In The Liberator Garrison argued that the verdict relied on "circumstantial evidence of the most flimsy character ..." and feared that the determination of the government to uphold its decision to execute Goode was based on race. As all other death sentences since 1836 in Boston had been commuted, Garrison concluded that Goode would be the last person executed in Boston for a capital offense writing, "Let it not be said that the last man Massachusetts bore to hang was a colored man!" Despite the efforts of Garrison and many other prominent figures of the time, Goode was hanged on May 25, 1849.



Garrison became famous as one of the most articulate, as well as most radical, opponents of slavery. His approach to emancipation stressed "moral suasion," non-violence, and passive resistance. While some other abolitionists of the time favored gradual emancipation, Garrison argued for "immediate and complete emancipation of all slaves." On July 4, 1854, he publicly burned a copy of the Constitution, condemning it as "a Covenant with Death, an Agreement with Hell," referring to the compromise that had written slavery into the Constitution.[17] In 1855, his eight-year alliance with Frederick Douglass disintegrated when Douglass converted to political abolitionists' view that the document could be interpreted as being anti-slavery.


Women also played an important role in the abolitionist effort. Harriet Tubman, born in Dorchester County, in Maryland in 1821, she was a fugitive slave and abolitionist who became an important figure of the Underground Railroad. A devoted Christian who prayed to God for her strength and guidance, Harriet Tubman became a friend of many of the best known Abolitionists and their sympathizers: Ralph Waldo Emerson, William Seward, Thomas Wentworth Higginson, Lydia Maria Child and Wendell Phillips. John Brown refers to her in his letters as one of the best and bravest persons on this continent - General Tubman as we call her.(Encyclopedia Britannica, 1978 Vol. 10, p.167).

Born to slave, parents, she escaped to freedom in 1849 by following the north star. As Tubman traveled North, she was aided along the way by people who part of the Underground Railroad, these stops came to be known as 'stations'. Throughout the 1850's she became one the Underground Railroad's most active 'conductor' making repeated journeys into slave territory, and leading about 300 other fugitives, including her parents, to freedom.

When the Fugitive Slave Law was passed in 1850, she extended the journey to freedom beyond the Canadian border to insure the safety of her 'passengers'. In guiding over 300 runaway slaves to freedom she became known as the, 'Moses of her people'. Maintaining the highest discipline on flights North, Tubman often forced panicky or exhausted 'passengers' ahead by threatening them with a loaded pistol.






When the Civil War began in 1861, she served as an army cook and nurse and later became a spy and guide for Union raids into Maryland and Virginia. After the war she managed a home in Auburn, New York, for indigent and elderly blacks until her death on March 10th, 1913; she was buried with full military honors.

Susan Brownell Anthony was born and raised in Adams, Massachusetts on February 15th, 1820, she was one of the greatest icons of women's rights, and was the daughter of a Quaker abolitionist. After completing her education in New York, she accepted employment as a teacher. Unsatisfied with the choice of profession, she went to work at the position of assistant manager of the family farm in upstate New York.

Her work proved rewarding, because it gave her the opportunity to meet and to discuss with some of her father's guests, and absorb some of the nature of American reform. Exposure to the views of such men as William Lloyd Garrison, Wendell Phillips, and Frederick Douglass convinced her that she, too, could become an advocate of reform. She worked as an agent for the Daughters of Temperance and for the American Anti-Slavery Society from 1856 until the outbreak of the Civil War. When she later joined up with Elizabeth Cady Stanton she published the New York liberal weekly The Revolution from 1868 to 1870 which demanded equal civil and political rights for women and blacks under the fourteenth and fifteenth amendments.

Harriet Elizabeth Beecher Stowe, author of Uncle Tom's Cabin, an antislavery novel of such force that it is often shown as one of the causes of the Civil War. The effect of Uncle Tom's Cabin on the conscience of northerners who read it was so powerful, that Abraham Lincoln called her 'the little women that started the great war.' Stowe, born in Litchfield, Connecticut on June 14, 1811, also wrote excellent depictions of rural New England life. Long overshadowed by her more sensational work, The Minister's Wooing and Oldtown Folks have recently gained appreciative audiences, and scholars and critics have begun to recognize that Uncle Tom's Cabin contains nearly as much art as propaganda.

The daughter of a celebrated Congregationalist minister, Lyman Beecher, Harriet relocated to Cincinnati at the age of 21, where she met and married Calvin Ellis Stowe, a biblical scholar. Her first publication, The Mayflower, revealed her interest in New England personalities, but her proximity to Kentucky had also given her firsthand knowledge of the South. When she and her husband moved to Brunswick, Maine, in 1850, she drew upon her recollections to write Uncle Tom's Cabin, followed by The Key to Uncle Tom's Cabin and Dred: A Tale of the Great Dismal Swamp, all of which originated in her lifelong hatred of slavery. After the Civil War, she continued to write essays, novels, and poetry, returning to the New England scenes with which she had begun her career. By the time she died, Stowe had long been recognized at home and abroad as one of America's foremost literary celebrities.



These abolitionists, as well as many others, made great strides in the struggle to end slavery and injustice in America. It was a long difficult battle for equality among men that took many years and a war to make southerners finally acknowledge that slavery was not only wrong and unjust but hypocritical to the Declaration of Independence.	



Works Cited Page 



Harriet Tubman." Bio.com. A&E Networks Television, n.d. Web. 10 May 2016.
John Brown." Bio.com. A&E Networks Television, n.d. Web. 10 May 2016
"John Brown Abolitionist." JOHN BROWN History. History Organization, n.d. Web. 10 May 2016 


